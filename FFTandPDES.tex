\documentclass{article}
\usepackage[utf8x]{inputenc}
\usepackage{default}
\usepackage{graphicx}
\usepackage[sort&compress]{natbib}

\def\r{{\bf r}}
\def\rp{{\bf r^\prime}}
\def\k{{\bf k}}
\def\q{{\bf q}}
\def\G{{\bf G}}
\def\R{{\bf R}}
\def\Gp{{\bf G^\prime}}
\def\sig{$\Sigma$}
\def\inveps{\varepsilon^{-1}}
\def\eps{\varepsilon}
\def\symm{\left\{\mathcal{S}|\mathbf{v}\right\}}
\def\S{\mathcal{S}}
\def\rt{\tilde{r}}
\def\pt{\tilde{p}}
\def\T{\hat{T}}
\def\bra{\langle}
\def\ket{\rangle}
\def\field{\hat{\psi}}
\def\cfield{\hat{\psi}^{\dagger}}
\def\I{\hat{I}}
\def\E{\varepsilon}
\def\H{\hat{H}}
\def\v{\mathbf{v}}
\def\Gs{\mathcal{G}}
\def\P{\hat{P}_{\rm occ}}

\begin{document}
Fourier Transforms do them all once (or better yet three times), remember them forever. If you forget. Look them up here:
\section{Gaussian Function}
%
\begin{equation}
f(\omega) = \int_{-\infty}^{+\infty} e^{\frac{-t^{2}}{\sigma}} e^{-i\omega t} dt
\end{equation}
%
\begin{equation}
f(\omega) = \int_{-\infty}^{+\infty} e^{-(\frac{t^{2}}{\sigma} + i\omega t)} dt
\end{equation}
%
\begin{equation}
f(\omega) = \int_{-\infty}^{+\infty} e^{-(\frac{t^{2}}{\sigma} + i\omega t)} dt
\end{equation}
%
Complete the square of the argument in the exponential to obtain:

%
\begin{equation}
f(\omega) = e^{-\frac{\omega^{2} \sigma}{4}} \int_{-\infty}^{+\infty} e^{-(\frac{t}{\sqrt{\sigma}} + \frac{i\omega \sqrt{\sigma}}{2})^{2}} dt
\end{equation}
%
The final step involves a change of variables $x = \frac{t}{\sqrt{\sigma}} + \frac{i\omega\sqrt{\sigma}}{2}$.
%
\begin{equation}
f(\omega) = \sqrt{\sigma}e^{-\frac{\omega^{2} \sigma}{4}} \int_{-\infty}^{+\infty} e^{-x^{2}} dt
\end{equation}
%
\begin{equation}
f(\omega) = \sqrt{\sigma \pi} e^{-\frac{\omega^{2} \sigma}{4}} 
\end{equation}
%
Why is this change of variables permissible? Consider a plot of the Gaussian in the entire complex plane. If the imaginary part in our exponential
goes to zero we recover a normal Gaussian function. As the imaginary part increases the center of this Gaussian is displaced into the complex plane.
Though, now, for purely real arguments the function has an oscillatory behaviour, which might be difficult to integrate,
the change of variables allows us to shift the function back onto the real line and evaluate the integral directly.
%
For large $\sigma$ we have a very narrow distribution in $\omega$.

\section{Sawtooth Wave Function}
%
To obtain the generic Fourier transform of the Sawtooth Wave Function we will 
first consider the simplest case. Then scale and transform the resulting
Fourier series to obtain a saw tooth for generic cases.
The function f(x) will move range from $[-1:1]$ linearly across
the domain $[-\pi, \pi]$. The function is \emph{odd} and we
will therefore only consider $b_{n}\sin{(nx)}$ components: 

The integral is then:
%
\begin{equation}
b_{n} = \frac{1}{\pi} \int_{-\pi}^{\pi} \frac{x}{\pi} \sin(nx) dx
\end{equation}
%
A single integration by parts gives:
%
\begin{equation}
b_{n} = \frac{1}{\pi^{2}}\left[\frac{-[x \cos(nx)]_{-\pi}^{+\pi}}{n} + \int_{-\pi}^{+\pi} \frac{\cos(nx)}{n} dx\right]
\end{equation}
%
The second term is zero. The first term when expanded yields the sum of two cosine functions with 
an argument $n\pi$. Since n is an integer the sum of the cosine terms will be either positive
or negative depending on whether n is odd or even. The final expression for the $b_{n}$ coefficients
can now be written:
%
\begin{equation}
b_{n} = \frac{2(-1)^{n+1}}{n \pi}
\end{equation}
%
So the Fourier series for the sawtooth becomes:
%
\begin{equation}
f(x) = \sum_{n=1}^{\infty} (\frac{2(-1)^{n+1}}{n\pi})\sin(nx)
\end{equation}
%

For functions with a frequency $f$ a single cycle is completed in the time $T$.
This is just the identity of the frequency and the period $fT=1$. In order for
our $sin$ function to complete an entire oscillation in the same period let
us replace its argument ($nx$) with ($2\pi f t$). Similarly the function's
amplitude can be scaled by $A$ and the function can be displaced along the
y-axis.
%
\begin{equation}
f(t) = c + \sum_{n=1}^{\infty} \frac{2(-1)^{n+1}}{n\pi}\sin(2\pi n ft)
\end{equation}
%
\bibliographystyle{unsrtnat}
\scriptsize
\bibliography{/users2/lambert/Documents/Bibliography/bibliography.bib}
\end{document}
